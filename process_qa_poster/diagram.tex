\documentclass{standalone}

\usepackage{tikz}
\usetikzlibrary{shapes,arrows}
%\usepackage{verbatim}

\begin{document}


\tikzstyle{srlb} = [draw, fill=orange!20, rectangle, 
    minimum height=3em, minimum width=6em]

\tikzstyle{mapperb} = [draw, fill=gray!20, rectangle, 
    minimum height=3em, minimum width=6em]
    
\tikzstyle{alignb} = [draw, fill=purple!20, rectangle, 
    minimum height=5em, minimum width=6em, align=center]

\tikzstyle{entailb} = [draw, fill=blue!20, rectangle, 
    minimum height=3em, minimum width=5em]

\tikzstyle{processkbb} = [draw, fill=red!20, rectangle, 
    minimum height=6em, minimum width=4em]
    
    
\tikzstyle{pinstyle} = [pin edge={to-,thin,black}]

\begin{tikzpicture}[auto, node distance=2cm,>=latex']
    \node [srlb, pin={[pinstyle]Question}] (srl1) {SRL};
	\node [right of=srl1, node distance=2cm](dummy1) {}; 
	\node [mapperb, right of=srl1, pin={[pinstyle]Answer Choices},
            node distance=4cm] (srl2) {Mapper};
    \node [processkbb, right of=srl2, node distance=3cm] (processkb) {Process KB};

    \node [alignb, below of=dummy1, node distance=2.5cm] (align) {Alignment\\and\\Ranking};

    \node [entailb, right of=align, node distance=5cm] (entail) {Entailment};
    \node [below of=align] (dummy2) {Predicted Answer};
    
    \draw [->] (srl1) -- ++(0,-1.3) node(lowerright){} -- node[above] {$Q$} ++(1.3,0) node(lowerright){} -- ++(0,-0.3) node {} (align);
    \draw [->] (srl2) -- ++(0,-1.3) node(lowerright){} -- node[above] {$A_M^N$} ++(-1.3,0) node(lowerright){} -- ++(0,-0.3)  node {} (align);
    \draw [<->] (srl2) -- node {} (processkb);
    \draw [<->] (align) -- node {$entails(x, y)$} (entail);
    \draw [->] (align) -- node {} (dummy2);
%\end{tikzpicture}
	% We start by placing the blocks
%    \node [input, name=input] {};
%    \node [sum, right of=input] (sum) {};
%    \node [block, right of=sum] (controller) {Controller};
%    \node [block, right of=controller, pin={[pinstyle]above:Disturbances},
%            node distance=3cm] (system) {System};
%    % We draw an edge between the controller and system block to 
%    % calculate the coordinate u. We need it to place the measurement block. 
%    \draw [->] (controller) -- node[name=u] {$u$} (system);
%    \node [output, right of=system] (output) {};
%    \node [block, below of=u] (measurements) {Measurements};
%
%
%    % Once the nodes are placed, connecting them is easy. 
%    \draw [draw,->] (input) -- node {$r$} (sum);
%    \draw [->] (sum) -- node {$e$} (controller);
%    \draw [->] (system) -- node [name=y] {$y$}(output);
%    \draw [->] (y) |- (measurements);
%    \draw [->] (measurements) -| node[pos=0.99] {$-$} 
%        node [near end] {$y_m$} (sum);
\end{tikzpicture}

\end{document}